\begin{abstract}
\label{sec:abstract}
This work proposes estimating sky radiance distribution curves between 350-2500nm from images captured with a hemispherical digital camera. A novel hardware system simultaneously captured spectral, spatial, and temporal information to acquire accurate physical measurements of the solar/skydome radiance variation. To achieve this goal, we use a custom-built spectral radiance measurement scanner to measure the directional spectral radiance, a pyranometer to measure the irradiance of the entire hemisphere, and a commercial digital camera to capture high-dynamic range (HDR) hemispherical imagery of the sky. We use the measurements obtained from a commercially available digital camera and correlating spectroradiometer measurements to train machine learning (ML) models to estimate whole sky full-spectrum radiance distributions (VIS, VNIR, and SWIR) from a low dimensional RGB input. We train clear, cloudy, and mixed sky models, and cross-validate the estimated radiance distributions with ground-truth data. We highlight important measured and engineered ML features, and we present useful feature engineering techniques employed to minimize model estimation error. Additional contributions of this work include the code for all ML models and experiments, a dataset of all-sky HDR captures with correlating spectroradiometer measurements captured 453 times over 16 days, and an open-source, cross-platform, interactive viewer used to visualize photometric and radiometric data side by side.
\end{abstract}