\section{INTRODUCTION}
\label{sec:introduction}

Multispectral sky radiance is needed for wavelength dependent computations in a wide variety of domains and applications, including: radiative transfer,\cite{chandrasekhar_radiative} building-performance simulation,\textsuperscript{\citenum{jakica_survey},\citenum{hensen_buildingperformance}} photovoltaic (PV) panel alignment\cite{smith_tilt}, climate science,\cite{lopez-alvarez_using_2008} physically based rendering\textsuperscript{\citenum{jakob_mitsuba},\citenum{haber_physically}} natural light transport,\textsuperscript{\citenum{veach_metropolis},\citenum{jarosz_montecarlo}} and any thermal/lighting application using bidirectional reflectance distribution functions (BRDFs).\cite{kurt_survey, cook_torrance_brdf, deering_atmospheric, nicodemus_brdf} The spectra ranges of interest can vary based on a number of factors; we use the following: visible (VIS) (380-780nm), visible and near-infrared (VNIR) (380-1400nm), and short wave infrared (SWIR) (1000-2500nm). Of even more importance is the need for collections of whole sky angular dependent radiance distribution curves required for accurate computations on surfaces affected by the angle of incidence, e.g. walls, facades, PV panels, all of which require more than just a single downward-welling radiance measurement, and all of which need to account for occlusions of the visible skydome.\cite{schumann_environment} Despite the fact that skylight itself has been studied for well over one hundred years (going back to Lord Rayleigh\cite{strutt_lightfromsky_1871}, Mie\cite{mie_beitrage_1908}, Kimball\cite{kimball_intensity}, and Pokrowski\cite{pokrowski_1929}, to name a few), the process of actively measuring or even computing whole sky radiance distributions quickly and accurately still remains an open problem, especially if the curves are to be used in a real-time, practical setting. Reliable, fast, cost-effective methods are still desired. This work attempts to use various modern machine learning methods to train models with the express purpose of being used in a real-time setting to estimate whole sky radiance curves quickly, within some acceptable error, using a dataset of whole sky images and radiance distributions measured with a custom-built hardware system.\cite{kider_framework_2014}

% \begin{figure}[btp]
% \begin{center}
% \includegraphics[width=1.0\textwidth,height=0.25\textwidth]{img/intro_vns.jpg}
% \end{center}
% \caption{\label{fig:spectrum}Sky radiance distribution spectra of interest: VIS, VNIR, and SWIR.}
% \end{figure}

% Some relevant contributions to this problem in the last decade include the following: Refs. \citenum{saito_estimation_2016, chauvin_modelling_2015, kocifaj_unified_2015, tohsing_validation_2014, cordero_downwelling_2013, roman_calibration_2012, ehrlich_airborne_2012, pissulla_comparison_2009, lopez-alvarez_using_2008, cazorla_using_2008, cazorla_development_2008, lee_jr_measuring_2008, milton_estimating_2006}.

Saito et al. estimated sky radiance distributions with a derived equation taking total ozone column and RAW RGB counts.\cite{saito_estimation_2016} They specifically attempt sky radiance estimation \enquote{without any training sets,}\cite{saito_estimation_2016} but focus on a single point in the sky (the zenith) for a subset of visible wavelengths (430-680nm). An important contribution is their camera-specific color matching functions (CMFs), an extension of work done by Sigernes et al.,\cite{sigernes_sensitivity} which take into account camera lens wavelength dependence, vignetting effect, and CMOS noise.\cite{saito_estimation_2016} 

Tohsing et al. used a machine learning approach to estimate full visible spectrum whole sky radiance distributions. They used non-linear regressions, one per color component (RGB) per wavelength (380-760nm), or 1143 models, using the non-linear relationships between RGB counts and radiance values.\cite{tohsing_validation_2014} Although they use 113 sampled points of the hemisphere, their measurements are limited to 8 of 12 consecutive days of the year, with only 1 day's clear sky data and 1 day's overcast data used for training.\cite{tohsing_validation_2014} They classify sky cover procedurally into 2 groups (clear or cloudy) via sky index,\textsuperscript{\citenum{yamashita_cloud},\citenum{saito_cloud}} and train 2 separate groups of regressions; scattered skies thus use one or the other. Because full hemisphere measurements took 12 minutes to complete, a synchronized, synthetic image was constructed and sampled.

Chauvin et al. also used a custom-built sky viewer/imager to obtain and then remove/clean anisotropy from clear sky portions of sky images for better cloud detection and ultimately better solar plant control.\cite{chauvin_modelling_2015} As stated by the authors, they focus on irradiance/intensity as opposed to radiance curves. Conversely to many sky models, they build upon the work of Grether et al. and Buie et al., noting the importance of the circumsolar region and its ratios along with the central angle from sun position to point position, sun-to-point angle (SPA), and explain how it can be used to estimate clear sky radiance.\cite{chauvin_modelling_2015}

Cazorla, Olmo, L\'{o}pez-\'{A}lvarez, and Alados-Arboledas, et al. used a variety of machine learning techniques, including artificial neural networks (ANN), genetic algorithms (GA) and pseudoinverse linear regression, in a series of projects using images from their own custom-built sky viewer.\cite{lopez-alvarez_using_2008, cazorla_using_2008, cazorla_development_2008} The ANN was used to perform cloud detection using color features extracted from sky images while the GA was used to optimize/minimize these features for the input layer of the network.\cite{cazorla_development_2008} A pseudoinverse linear regression model was used on a datatset of 902 samples of sky image RGB values and visible spectrum radiance at $45\degree$, $60\degree$, and $90\degree$ zenith angles. The final training set is whittled down to 40 samples.\cite{cazorla_development_2008}

Kocifaj's model, while well researched and presented, might be too computationally expensive for a real-time application, and is therefore tangential to this project.\cite{kocifaj_angular_2012, kocifaj_unified_2015} Cordero et al. studied albedo effect on radiance distributions (both upwelling and downwelling).\cite{cordero_downwelling_2013} Lee studied overcast skies to find meridional consistencies.\cite{lee_jr_measuring_2008} Pissulla et al. compare 5 separate spectroradiometers and present their measurement deviations.\cite{pissulla_comparison_2009} %Juan and Da-Ren present 3 (geometric, optical, and radiometric) calibration experiments on a sky viewer/imager.\cite{juan_calibration_2009}

A discussion on sky radiance distributions is not possible without acknowledging turbidity and its attenuating effects on terrestrial solar radiance. Rayleigh,\cite{strutt_lightfromsky_1871} Mie,\cite{wriedt_miereview_2012, wriedt_miebook_2012, mie_beitrage_1908} and non-selective scattering algorithms explain it. Although it must be accounted for in models of highest accuracy, it can sometimes be ignored or roughly estimated. There is a wealth of literature in the atmospheric and solar energy communities that highlight the consistency of the ``cosine factor" of solar zenith angle (SZA) on radiance curves, starting with Steven and Unsworth in the mid-1970s, who recorded standard radiance distributions of clear skies, and in doing so found \enquote{departures due to variation in atmospheric turbidity [...] to be small}.\cite{steven_standard_1977}

% This paper is organized as follows. In Section \ref{sec:datacapture} we explain our data capture hardware (\ref{sec:hardware}), measurements (\ref{sec:measurements}), and observed data (\ref{sec:data}). In Section \ref{sec:methods} we explain our methods and experiments for spectral shape estimation. Results are presented in Section \ref{sec:results} with concluding remarks in Section \ref{sec:conclusions}.